\documentclass[11pt,letter]{article}
\usepackage[top=1.00in, bottom=1.0in, left=1.1in, right=1.1in]{geometry}
\usepackage{graphicx} % Required for inserting images

\title{Solstice optimizes thermal growing season}

\author{Victor, Lizzie}
\date{Aug.-Nov. 2024}

\begin{document}

\maketitle

%\begin{enumerate}

%\item New (recent) exciting findings! Solstice, "\emph{celestial starting gun}"\\
%(Bad) consequences for climate change forecasting: plants are stuck? \\
% $\rightarrow$ But leave an open question: why solstice?
Plants generally used environmental cues, such as temperature and photoperiod, to adjust the timing of phenological events in response to environment variability within and between years. Yet, the larger mechanisms behind these cues remain largely unknown for many events. 

Recently, summer solstice has been proposed as a universal trigger to modulate cues and initiate key physiological processes \cite{Zohner2023, Journe2024}. This proposed photoperiod switch, if correct, could reshape predictions of forest responses to climate change. However, using a fixed date like the solstice as a cue could limit plasticity and become less suitable in a warmer future climate \cite{Wolkovich2021}. 
This may in turn drive forest declines, with significant implications for carbon storage, and thus raises important questions about the suitability for plants to rely on the solstice. 

% \item Plants need to balance transition to events at best moment vs end-of-season constraint, i.e. choose when to transition without full info\\
% $\rightarrow$ Need to optimize predictions!
The timing of major plant phenological transitions---such as start of growth with leafout---should match development states with fitness opportunities, given no other constraints. In most environments, this will always involve a trade-off between an increased fitness (e.g. a longer window for growth) and an increased susceptibility to climatic and biotic risks (e.g. a higher exposition to late frosts). Because plants cannot know these exact opportunities and risks in advance, they should rely on the most informative cues to accurately anticipate environmental conditions and optimize their chances for growth and reproduction \cite{Chevin2015, Bonamour2019}.

% \item (Thermal) $GDD$ as a critical integrator\\
% Critical both to crops and wild plants\\
%Plants want to grow as much as possible during warm years, and benefit from warm years to set many flowers (flower differentiation?)
In particular, it is well established that plants respond primarily to integrated climate forcing, often measured as the accumulation of temperatures---in a given range where metabolism is sufficient---over a given period (growing degree-days, GDD). This heat accumulation is a key factor in development and growth processes of both crops (REF) and wild plants (REF). The number of GDD accumulated throughout the season directly impacts how quickly cells elongate to form new organs and how quickly a plant progresses through growth stages. Plants are thus expected to take full advantage of warmer years (with high GDD) to maximize growth and set many flowers for the following season.

% \item Given $GDD$ is so important, plants should thus transition into events when they can best predict $GDD_{total}$ within growing season -- while still having enough time/energy \emph{to do what they need to do}
Given the importance of GDD, plants should ideally time their transitions when their ability to predict the total GDD within the growing season is maximum while still having enough potential thermal energy to complete essential growth and reproductive processes. Concretely, this trade-off means that there should be an optimal period during which the plant has accumulated enough GDD to reliably predict the total GDD by the end of the year (\emph{environmental predictability}), while also maximizing the remaining GDD available (\emph{growth potential}). Here, we define the environmental predictability at a day $d$ as the $R^2$ of a linear regression, across years, between the total GDD (that will be accumulated at the end of the year) and the GDD already accumulated between 1st January and $d$. This simple definition allows us to examine which window in the season appears optimal for the plant to maximize its growth and development while minimizing risks.

% \item We found this optimal point in Europe to be the solstice(ish)\\
% \begin{enumerate}
%     \item If this true widely, then evolution towards an universal solstice trigger could make sense, right? 
%     \item But our results also suggest we cannot teasee appart solstice and thermal optimum-predictability cue
% \end{enumerate}
Globally, we found this optimal point to be around the summer solstice in Europe (Fig). If this specific day indeed represents a broad-scale optimum across different climatic conditions, genetic evolution towards a universal solstice trigger could make sense---especially since this optimum was stable over the Holocene (supp Fig). Yet, our results also suggest that we cannot disentangle the influence of the solstice from that of a thermal optimum cue. This could indicate that plants rely on a combination of both solstice and thermal cues to optimize growth and reproductive timing. Alternatively, this overlap could simply be a coincidence, since it would be costly for plants to closely track two different signals---i.e. to encode and decode both information within their cells. 
% a critical juncture

% \item Supporting this we found important variation over space (and time?)
Supporting this "cautionary" hypothesis, we found important variation across locations in Europe. Obviously, ...


% \item This means:
% \begin{enumerate}
%     \item critical need for experiment to disentangle daylength vs temp.
%     \item researchers need to more clearly test for (i) trends estimated at local scale across gradient and (ii) how they scale up to (sub)continental/global scales
% \end{enumerate}

%\end{enumerate}

\end{document}
