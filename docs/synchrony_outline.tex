\documentclass[11pt,letter]{article}
\usepackage[top=1.00in, bottom=1.0in, left=1.1in, right=1.1in]{geometry}
\usepackage{graphicx} % Required for inserting images

\title{Solstice optimizes thermal growing season}

\author{Victor, Lizzie}
\date{August 2024}

\begin{document}

\maketitle

\section*{Outline}

\begin{enumerate}

\item New (recent) exciting findings! Solstice, "\emph{celestial starting gun}"\\
(Bad) consequences for climate change forecasting: plants are stuck? \\
$\rightarrow$ But leave an open question: why solstice?

\item Plants need to balance transition to events at best moment vs end-of-season constraint, i.e. choose when to transition without full info\\
$\rightarrow$ Need to optimize predictions!

\item (Thermal) $GDD$ as a critical integrator\\
Critical both to crops and wild plants\\
Plants want to grow as much as possible during warm years, and benefit from warm years to set many flowers (flower differentiation?)

\item Given $GDD$ is so important, plants should thus transition into events when they can best predict $GDD_{total}$ within growing season -- while still having enough time/energy \emph{to do what they need to do}

\item We found this optimal point in Europe to be the solstice(ish)\\
\begin{enumerate}
    \item If this true widely, then evolution towards an universal solstice trigger could make sense, right? 
    \item But our results also suggest we cannot teasee appart solstice and thermal optimum-predictability cue
\end{enumerate}

\item Supporting this we found important variation over space (and time?)

\item This means:
\begin{enumerate}
    \item critical need for experiment to disentangle daylength vs temp.
    \item researcher need to more clearly test for (i) trends estimated at local scale across gradient and (ii) how they scale up to (sub)continental/global scales
\end{enumerate}

\end{enumerate}

\end{document}
